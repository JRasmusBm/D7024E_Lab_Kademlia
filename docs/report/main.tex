\documentclass[a4paper]{article}

\usepackage{minted} % Display Code
\usemintedstyle{friendly}
\setmintedinline{breaklines}
\usepackage{url} % Show URLs
\usepackage[utf8]{inputenc} % Encoding
\usepackage{chngcntr} % Counters
\counterwithin{figure}{section}
\usepackage{hyperref} % References
\usepackage{graphicx}

\renewcommand{\abstractname}{Executive Summary}

\title{Kademlia\\ \medskip{}
  \large{Sprint 0}
}

\author{
  August Eriksson~\thanks{augeri-5@student.ltu.se}\\
  Rasmus Bergström~\thanks{rasber-5@student.ltu.se}\\
  Robin Rubindal~\thanks{robrub-5@student.ltu.se}\\
}

\begin{document}

\maketitle
\newpage

\begin{abstract}
This covers the project of a course in distributed systems (D7024E)
given at Luleå Tekniska Universitet (LTU). The students are introduced
to a distributed system principle (Kademlia) which is used in practice
by Ethereum, BitTorrent, IPFS and more. Kademlia is to be implemented
throughout this project in Google-go as well as Docker. During the first
week it was concluded that all mandatory objectives is to be completed
during the first sprint, this includes a CLI that supports the following
commands get (retrieve
object), put (upload object) and exit (terminate node) as well RPCs for
lookup node, pinging, network joining, store value, find value and 50\%
test coverage. At the end of the first sprint most of the mandatory
objectives were complete or the majority of the objectives had been
fulfilled. For the second sprint it was concluded that any remaining
mandatory objectives were to be fulfilled and all qualifying objectives
except for concurrency were to be fulfilled. This includes a RESTful api
for future web application intergration as well as time-to-live (TTL)
functionality to allow objects to silently be deleted if instructed to
stop refreshing the TTL and at least 80\% test coverage. At the end of 
sprint 2 all mandatory objectives were completed and almost a 100\% test
coverage was reached. However the RESTful api and TTL functionality had
to be reconsidered and was not implemented during this project.
\end{abstract}


\newpage

\section{Introduction}
This is the project lab in the course D7024E taught at Luleå Tekniska Universitet (LTU), in which the students get to work with the distributed hash table (DHT) Kademlia by implementing a kademlia solution using containers to spin up a network. This project covers an implementation using google-go and docker.

\subsection{Architecture}
Below is a figure of the general architecture of a node before implementation, this will be updated during implementation.
\begin{figure}[ht]
\centering
\includegraphics[width=\linewidth]{D7024E-architecture.jpg}
\caption{High-level architecture of a node in the kademlia implementation.}
\end{figure}

The main function starts the reciever listener as the 'busy-wait' loop so that the node does not terminate but awaits further instructions. The receiver then spawns new threads (goroutines) as needed to handle incoming instructions. The sender is used to send any messages to other nodes in the network. Utils handles any general functions such as hashing, a node getting its own IP, etc. While the routing table keeps records of a nodes k-buckets.

\newpage

\section{Sprint 1 Plan}
After a discussion we estimated that we will complete the mandatory objectives for this project during the first sprint. For the backlog, see our Github.\footnote{\href{https://github.com/JRasmusBm/D7024E_Lab_Kademlia}{https://github.com/JRasmusBm/D7024E\_Lab\_Kademlia}}\\

\textbf{Networking:} A node shall be able to ping any other node, a node should be able to join the Kademlia network and a node shall be able to find any other node in the Kademlia network.\\

\textbf{Objects:} Any node must be able to upload an object (key:val pair) that will be stored on appropriate nodes according to the kademlia principle. Any node must be able to find and download any object stored on at least one node of the kademlia network.\\

\textbf{CLI:} Each node must be able to execute the following commands: put, get and exit. Exit terminates the node, get takes a hash as its arguement and outputs the node and the value sought if it exists. Put takes a value as an arguement and outputs the hash if uploaded successfully.\\

\textbf{Testing:} At least a 50\% testing coverage of the implementation.\\

\textbf{Containers:} A network of at least 50 nodes must be spinned up by containers where each node is a seperate container.\\

\bibliographystyle{plain}
\bibliography{sources}

\end{document}
