\section{Planning}\label{sec:planning}

\subsection{Sprint 1}\label{sec:sprint-1}
After a discussion we estimated that we will complete the mandatory objectives
for this project during the first sprint. For the backlog, see our Github.
\footnote{\href{https://github.com/JRasmusBm/D7024E_Lab_Kademlia}{https://github.com/JRasmusBm/D7024E\_Lab\_Kademlia}}\\

\textbf{Networking:} A node shall be able to ping any other node,
a node should be able to join the Kademlia network and a node shall
be able to find any other node in the Kademlia network.\\

\textbf{Objects:} Any node must be able to upload an object
(key:val pair) that will be stored on appropriate nodes according
to the kademlia principle. Any node must be able to find and
download any object stored on at least one node of the kademlia network.\\

\textbf{CLI:} Each node must be able to execute the following
commands: put, get and exit. Exit terminates the node, get
takes a hash as its arguement and outputs the node and the value
sought if it exists. Put takes a value as an arguement and
outputs the hash if uploaded successfully.\\

\textbf{Testing:} At least a 50\% testing coverage of the implementation.\\

\textbf{Containers:} A network of at least 50 nodes must be spinned
up by containers where each node is a seperate container.

\subsubsection{Sprint 1 reflections}\label{sec:sprint-1-reflections}
Not all planned objectives were fulfilled at the end of sprint 1,
but a good foundation has been developed going into sprint 2.\\

\textbf{Containers:} We have a network of 50 nodes spinned up by
containers and each node is in a seperate container. This objective
is complete.\\

\textbf{Testing:} We are continously adding tests to our code,
this objective should be complete before the official end of sprint 1.\\

\textbf{CLI:} We have a simple shell that can connect and communicate
with a server in each node. The functionality itself is dependent on
the RPCs and not fully operational at this point.\\

\textbf{Objects and Networking:} These are our RPCs and require the
most work. They are beginning to become operational but are still
work in progress at this point.

\newpage

\subsection{Sprint 2}\label{sec:sprint-2}
The plan is to implement all qualifying objectives except for
concurrency after implementing any remaining mandatory objectives.
For the backlog, see our Github.
\footnote{\href{https://github.com/JRasmusBm/D7024E_Lab_Kademlia}
{https://github.com/JRasmusBm/D7024E\_Lab\_Kademlia}}\\

\textbf{Object expiration:} Implement Time-To-Live(TTL) functionality.
When TTL expires the data object is silently deleted. But the TTL is
reset every time the object is requested and transmitted.\\

\textbf{Object expiration delay:} To prevent objects from expiring,
the original node that uploaded the object sends a refresh command
to reset the TTL without actually requesting the object. As long
as the uploading node can contact the storing nodes, the object
should never expire.\\

\textbf{Forget CLI command:} The command takes the hashnr of the
object that is to be removed. (By stopping the refresh of TTL).\\

\textbf{RESTful application interface:} A restful HTTP interface
so it may be integrated with web applications or other applications
at some point. Implement POST-objects and GET-objects.\\

\textbf{Testing:} At least a 80\% test coverage of the implementation.

\subsubsection{Sprint 2 reflections}\label{sec:sprint-2-reflections}
We had to reconsider our plans throughout the sprint, realizing
that attempting to implement all planned functionality would
take a lot of time we decided to skip the implementation of a
RESTful api as well as any time-to-live(TTL) functionality.
Instead we focused on finishing any mandatory objectives as well
as refactoring a lot of the code to reach 100\%
test coverage in most cases to ensure a robust and good quality code.

Any skipped functionality has been mentioned in the section for
limitations and future work.
