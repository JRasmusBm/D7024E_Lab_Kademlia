\section{System description}\label{sec:system description}
\textbf{Cliserver:} The cliserver listens for connections from a client.
Once a client has connected it handles communication between the kademlia node
and the client. Entering an unsupported command results in a string asking the
client to enter another command. Currently supported commands are:

\begin{itemize}
  \item close: Closes the connection.
  \item exit: Closes the connection and terminates the kademlia node.
  \item help: Returns a list of supported commands.
  \item put filename: Uploads the contents of the file to the kademlia node and
attempts to store it in the kademlia network.
  \item get hashnr: Attempts to retrieve the value of the given hashnr if
it's stored in the kademlia network.
  \item ping ipaddr: The kademlia node attempts to ping the given IP address.
\end{itemize}

The commands are handled by passing them to an API between the cliserver and
the internal network of the node.\\\\
\textbf{Network:} The internal network of a kademlia node consists of a receiver
and a sender. The receiver will always be listening for new connections and to
handle their incoming requests while the sender will attempt to make connections
to nodes on the kademlia network and make requests. The following are the
implemented RPCs:

\begin{itemize}
  \item ping: Pings a known node to see if it is still alive.
  \item store: Attempts to store the hashed value on X nodes.
  \item findnode: Attempts to find a node in the kademlia network, if it's unknown
it requests the closest known node to find it.
  \item findvalue: Attempts to find the given value on the kademlia network.
  \item join: Attempts to join the kademlia network.
\end{itemize}

The kademlia network is encoded and decoded using json.\\\\
\textbf{Utils:} The utils package contains all utility functions such as hashing,
finding your own IP and more.\\\\
\textbf{Testing:} We strive towards 100\% text coverage and most if not all
previously mentioned system parts have 100\% coverage.
