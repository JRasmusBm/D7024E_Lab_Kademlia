\section{Summary}\label{sec:summary}
We chose to present a paper\footnote{\href{https://ieeexplore.ieee.org/
document/8457919}{https://ieeexplore.ieee.org/document/8457919}}
concerning blockchain-based e-voting systems. The problems this paper
attempts to solve is the core of any election. It should not enable 
coerced voting, a vote should not be traceable to a person, it should
provide transparency without sacrificing the voter's privacy. It should
provide a secure authentication, ensure only eligible individuals can vote
and should not allow any tampering by any third party.

The paper presents a solution using an Go-Ethereum permissioned blockchain and
mentions previous related work of attempted e-voting systems such as Agora.\footnote{
\href{https://static1.squarespace.com/static/5b0be2f4e2ccd12e7e8a9be9/t/5b6c38550e2e725e9cad3f18/1533818968655/Agora_Whitepaper.pdf}{https://static1.squarespace.com/static/5b0be2f4e2ccd12e7e8a9be9/t/5b6c38550e2e725e9cad3f18/1533818968655/Agora\_Whitepaper.pdf}} In the article they also cover the basics of
blockchain and why it is the base of the presented solution. The blockchain
is an immutable ledger that is easily verifiable that offers full transparency
while not sacrificing privacy. Their presented blockchain solution is based
on smart contracts, which they describe as "refedined trust" as they are
visible to all users of the blockchain and therefore easily verified.

The solution itself presents two types of nodes, a "district node" that
represents each voting district and interacts with a "bootnode". The
bootnode does not keep any blockchain state but helps district nodes discover
each other and communicate. Each institution with permission access will host
a bootnode.

In the article they defined the smart contracts as "Election roles",
"Agreement process" and "Transactions". There are two types of roles,
an "Election Administrator" that create an election and manage its lifecycle.
A "Voter" that are eligible individuals. A voter can authenticate, load election
ballots, vote and verify their vote.

There are 4 types of processes, the "Election creation" are smart contracts as
election ballots that are written onto the blockchain. "Voter registration" to
ensure only eligible individuals can vote, they mention that this step may need
a component for a government identity verification service. "Tallying results"
this is done on the fly in the smart contracts. "Verifying votes" in their solution
voters receive a transaction ID of their vote which is to be used on an official
election site to verify their vote.

The transaction itself is done by the voter interacting with a ballot smart contract.
The smart contract then interacts with the blockchain via its district node which
appends the vote onto the blockchain.

The limitation with this system presented in the article is that the system is
capable of handling hundreds of transactions per second. But for countries where
millions and millions of votes are done at the same time additional measures would
be needed to support such a nation scaled event.

\section{Reflections}\label{sec:reflections}
We thought it was an interesting concept of using blockchain as an e-voting system
and thus decided to present this paper. Todays traditional voting system has no transparency
so a voter is incapable of verifying their vote. Which is one of the major points these types
of solutions attempts to solve.
